\chapter{Funzioni della famiglia apply}
\section{Introduzione}

Molto spesso nelle pratiche applicazioni  al fine di condurre nel migliore dei modi la nostra analisi si rende necessario applicare una stessa funzione a una serie di elementi.

Il modo principale per ottenere ciò è attraverso la creazione di un apposito loop (una ripetizione consecutiva un numero di volte) con i costrutti di \erre\ a questo scopo preposti come \textsf{for}, \textsf{if}, eccetera.

Spesso però, con operazioni abbastanza frequenti (come la somma di tutti i valori di una colonna ad esempio) lo stesso risultato si può ottenere con una famiglia di funzioni appositamente costruita per sfruttare le caratteristiche proprie degli oggetti a cui si applicano ed ottenere lo stesso risultato di un loop ma con meno codice.

Queste funzioni appartengo alla cosiddetta famiglia \textsf{apply}.

\section{apply}
La prima di queste funzioni è \textsf{apply}. Questa funzione opera in maniera molto semplice permettendo con poco codice di applicare un'altra funzione, da specificare all'interno di \textsf{apply}, su una dimensione di un oggetto array e quindi anche sulla forma più basilare di array, un array bidimensionale o più volgarmente, una matrice.

Nella esecuzione non è più veloce di un loop (lo era nelle prime versioni del linguaggio \textsf{S}), è però molto più pratica in quanto come si è detto, richiede molto meno codice.

La sintassi è la seguente:
\begin{lstlisting}
apply(<array>, <dimensione>, <funzione>)
\end{lstlisting}

Dove con array abbiamo un oggetto ti tipo array anche multidimensionale ma anche un data frame, con funzione abbiamo il nome e soltanto il nome di una funzione da applicare e con dimensione abbiamo una delle dimensioni dell'oggetto.

La particolarità da ricordare onde evitare di incorrere in errori, è che l'oggetto \textsf{<funzione>} all'interno di \textsf{lapply} va inserito senza alcuna parentesi. In pratica va scritto solo il nome della funzione usata. Ad esempio se vogliamo la media, dobbiamo usare la funzione \textsf{mean} e scrivere solo il nome di quest'ultima.

Fondamentale è anche capire il concetto di dimensione. Banalmente una matrice (che è un array bidimensionale), ha appunto due dimensioni: righe e colonne. Ne consegue che quindi specificando 1 la funzione \textsf{<funzione>} all'interno di \textsf{<array>} sarà applicata alle righe mentre con 2 sarà applicata alle colonne.

Vediamo un esempio in cui creiamo un data frame usando due vettori di valori e successivamente applichiamo a quest'ultimo la funzione \textsf{sd} al fine di trovare la deviazione standard di ogni colonna.
\begin{lstlisting}
> valori1 <- c(1:5)
> valori1
[1] 1 2 3 4 5
> valori2 <- c(10:15)
> valori2
[1] 10 11 12 13 14
>
> dataF <- data.frame(valori1, valori2)
> dataF
  valori1 valori2
1       1      10
2       2      11
3       3      12
4       4      13
5       5      14
> apply(dataF, 2, sd)
 valori1  valori2 
1.581139 1.581139 
\end{lstlisting}

Alcune considerazioni sull'esempio precedente:
\begin{itemize}
\item la funzione \textsf{apply} si applica a data frame e array ed ha come output una lvettore di elementi; come si vedrà non sempre l'output è un oggetto vettore.
\item è stato utilizzato l'operatore \textsf{colon} dato dai due puntini \textsf{< : >} per creare una sequenza di numeri da 1 a 5 e da 10 a 14, quindi per ottenere due vettori di 5 elementi.
\item la lunghezza dei vettori da unire nel data frame deve essere uguale altrimenti \erre\ avrebbe restituito un errore.
\item il numero 2 usato come secondo elemento della funzione \textsf{apply} sta ad indicare la seconda dimensione del nostro oggetto e cioè le colonne. Sarebbe stato possibile calcolare la deviazione standard per ogni riga specificando la dimensione 1 e cioè righe.
\end{itemize}


\section{lapply e sapply}

La seconda di queste funzioni è \textsf{lapply}\footnote{È una funzione definita internamente in C.}, diminutivo di list apply. L'idea alla base di questa funzione è che data una lista di oggetti \textsf{lapply} applica una determinata funzione a tutti gli elementi di quella lista, quindi ad ogni singolo oggetto.

Questo oltre ad essere molto utile per oggetti che sono liste in senso stretto ha anche altre implicazioni. A ben ricordare un data frame non è altro che una list composta da vettori di eguali dimensioni.

La funzione lapply quindi prende ogni colonna del data frame, che in realtà non è altro che un elemento vettore a sua volta parte del data frame che è una list, e vi applica una specifica funzione. Ne consegue che è come se avessimo creato un loop per tutte le colonne (ma anche le righe) di un data frame.

La sintassi è molto semplice:
\begin{lstlisting}
lapply(<dataframe>, <funzione>)
\end{lstlisting}

Ecco un esempio in cui creiamo un data frame usando due vettori di valori e successivamente applichiamo a quest'ultimo la funzione \textsf{sd} al fine di trovare la deviazione standard di ogni colonna.

\begin{lstlisting}
> dataF
  valori1 valori2
1       1      10
2       2      11
3       3      12
4       4      13
5       5      14
>
> lapply(dataF, sd)
$valori1
[1] 1.581139

$valori2
[1] 1.581139
\end{lstlisting}

Come già detto, con \textsf{lapply} si ottiene come output un oggetto di classe list. Ciò a volte potrebbe non essere desiderabile e per ottenere un oggetto di tipo vettore, in \erre\ è implementata un'altra funzione che lavora esattamente come \textsf{lapply} ma semplifica il risultato offrendo come output un vettore. Questa funzione è \textsf{sapply}, che sta per simplified lapply.

Ecco un esempio usando il data frame creato in precedenza.

\begin{lstlisting}
> sapply(dataF, sd)
 valori1  valori2 
1.581139 1.581139 
\end{lstlisting}

Questa funzione opera dietro le quinte una semplificazione del risultato ottenuto, ma per far ciò deve cercare in qualche modo di capire quale potrebbe essere un risultato semplificato.

Questo processo che \textsf{sapply} esegue in background, con basi di dati di grosse dimensioni potrebbe essere abbastanza lento per i normali tempi di esecuzione e a volte anche per l'utente che potrebbe avvertire il lungo tempo di esecuzione.

Per ovviare a ciò potrebbe essere utile usare la funzione \textsf{vapply} che richiede un terzo argomento obbligatorio che indica il formato di output desiderato. In questo modo la computazione è molto più veloce perché la funzione sa già cosa deve restituite. Tuttavia \erre\ fa esattamente ciò che l'utente dice di fare a meno che non si tratti di qualcosa di impossibile, è bene quindi prestare attenzione a non richiedere un formato di output impossibile per i dati che \textsf{vapply} deve processare.

\section{tapply}
La funzione \textsf{tapply} è usata per applicare una funzione su sottoinsiemi di un oggetto matrice o data frame o array, divisi secondo particolari istruzioni definite da un altro sottoinsieme dello stesso oggetto\footnote{Ma anche di altri in effetti.}. 

Immaginiamo di avere un vettore numerico e di voler applicare una funzione a sottoinsiemi di questo vettore basati su alcuni caratteristiche definite da un altro vettore (che quasi sempre è un factor). In pratica \textsf{tapply} divide un vettore in pezzi più piccoli ed applica la funzione ad ognuno di questi.

È molto utile vedere un esempio. Supponiamo di misurare per un gruppo di sei persone, il loro sesso e il loro reddito. Al fine di condurre una ricerca su una eventuale differenza di reddito tra uomini e donne sarebbe utile con i nostri dati calcolare la media del reddito per ogni gruppo: uomini e donne.

Per prima cosa creiamo un data base e successivamente usiamo la funzione
\textsf{tapply} per applicare la funzione \textsf{mean} (media), sulla colonna reddito suddivisa però sulla base dei due livelli presenti nella colonna genere, quindi per sesso maschile e femminile.

\begin{lstlisting}
> d <- data.frame( list( genere = c("M","M","F","M","F","F"), 
reddito = c(55000,88000,32450,76500,123000,45650) ) )
> d
  genere reddito
1      M   55000
2      M   88000
3      F   32450
4      M   76500
5      F  123000
6      F   45650
> tapply(d$reddito, d$genere, mean)
       F        M 
67033.33 73166.67
\end{lstlisting}

L'output è composto dalle due medie della colonna reddito, una calcolata sulla base del fattore M e l'altra sulla base del fattore F.

\subsection{split}

Come abbiamo visto, \textsf{tapply} divide l'oggetto in gruppi ai quali applica separatamente una data funzione invocata dall'utente. La funzione \textsf{split} sostanzialmente si ferma al primo passo.

La funzione \textsf{split} divide un oggetto \erre\ sulla base delle indicazioni dell'utente e ha come output un oggetto list.

Nell'esempio sottostante si utilizza il data frame precedente, per ottenere due data frame entrambi elementi di un oggetto di livello superiore di class list.

\begin{lstlisting}	
> d1 <- split(d, d$genere)
> d1
$F
  genere reddito
3      F   32450
5      F  123000
6      F   45650

$M
  genere reddito
1      M   55000
2      M   88000
4      M   76500
\end{lstlisting}

\subsection{by}

Da un punto di vista concettuale la funzione \textsf{by} è esattamente come \textsf{tapply} con una piccola ma fondamentale differenza che la rende uno strumento potentissimo. Come si è visto la funzione \textsf{tapply} ha bisogno di due vettori per poter funzionare, dove il primo rappresenta i dati sui quali applicare la funzione ed il secondo i gruppi in base ai quali suddividere il primo vettore.

La funzione \textsf{by} permette di applicare una funzione ma non richiede che il primo elemento sia un vettore di uguali dimensioni ma accetta anche matrici, arrays e data frame. Per meglio comprendere le implicazioni di questa differenza ci serviremo di un esempio usando il database iris presente nel pacchetto \textsf{datasets}.

Supponiamo di voler calcolare per ogni fattore della colonna Species, le statistiche descrittive relative alla colonna Sepal.Width usando la utilissima funzione \textsf{summary}. Per questo semplice compito è possibile utilizzare sia la funzione \textsf{by} che \textsf{tapply}. Il codice seguente mostra l'uso di entrambe le funzioni, si noti che i risultati sono stati salvati in oggetti ai quai è stata applicata successivamente la funzione \textsf{class}. 

Si può notare come entrambi gli oggetti non siano atomic vectors ma strutture più complesse di classi differenti.  Questa è la prima differenza.
\begin{lstlisting}	
> ct <- tapply(iris$Sepal.Width, iris$Species, summary )
> class(ct)
[1] "array"
> is.atomic(ct)
[1] FALSE
> cb <- by(iris$Sepal.Width, iris$Species, summary )
> class(cb)
[1] "by"
> is.atomic(cb)
[1] FALSE
\end{lstlisting}

Stampando a video gli oggetti ct e cb che contengono i nostri output si può notare come, nonostante la classe differente, i risultati siano sostanzialmente gli stessi e, in questo caso, l'uso dell'una o dell'altra funzione può essere indifferente.

\begin{lstlisting}	
> ct
$setosa
   Min. 1st Qu.  Median    Mean 3rd Qu.    Max. 
  2.300   3.200   3.400   3.428   3.675   4.400 

$versicolor
   Min. 1st Qu.  Median    Mean 3rd Qu.    Max. 
  2.000   2.525   2.800   2.770   3.000   3.400 

$virginica
   Min. 1st Qu.  Median    Mean 3rd Qu.    Max. 
  2.200   2.800   3.000   2.974   3.175   3.800 

> cb
iris$Species: setosa
   Min. 1st Qu.  Median    Mean 3rd Qu.    Max. 
  2.300   3.200   3.400   3.428   3.675   4.400 
-------------------------------------------------------------- 
iris$Species: versicolor
   Min. 1st Qu.  Median    Mean 3rd Qu.    Max. 
  2.000   2.525   2.800   2.770   3.000   3.400 
-------------------------------------------------------------- 
iris$Species: virginica
   Min. 1st Qu.  Median    Mean 3rd Qu.    Max. 
  2.200   2.800   3.000   2.974   3.175   3.800 
>  
\end{lstlisting}

La vera differenza tra \textsf{tapply} e \textsf{by} entra in gioco quando l'elemento lungo il quale applicare la funzione non è più un vettore ma un data frame (ma anche array o matrice).

Il seguente codice interrompe la sua esecuzione e restituisce un messaggio di errore che ci informa che la dimensione dell'indice e dei dati sui quali applicare la funzione hanno dimensioni differenti e quindi \erre\ non sa come comportarsi.

\begin{lstlisting}	
> tapply(iris, iris$Species, summary )
Error in tapply(iris, iris$Species, summary) : 
  arguments must have same length
\end{lstlisting}

Proviamo invece esattamente la stessa sintassi utilizzando però la funzione \textsf{by} al posto di \textsf{tapply}.
\begin{lstlisting}	
> by(iris, iris$Species, summary )
iris$Species: setosa
  Sepal.Length    Sepal.Width     Petal.Length    Petal.Width          Species  
 Min.   :4.300   Min.   :2.300   Min.   :1.000   Min.   :0.100   setosa    :50  
 1st Qu.:4.800   1st Qu.:3.200   1st Qu.:1.400   1st Qu.:0.200   versicolor: 0  
 Median :5.000   Median :3.400   Median :1.500   Median :0.200   virginica : 0  
 Mean   :5.006   Mean   :3.428   Mean   :1.462   Mean   :0.246                  
 3rd Qu.:5.200   3rd Qu.:3.675   3rd Qu.:1.575   3rd Qu.:0.300                  
 Max.   :5.800   Max.   :4.400   Max.   :1.900   Max.   :0.600                  
-------------------------------------------------------------- 
iris$Species: versicolor
  Sepal.Length    Sepal.Width     Petal.Length   Petal.Width          Species  
 Min.   :4.900   Min.   :2.000   Min.   :3.00   Min.   :1.000   setosa    : 0  
 1st Qu.:5.600   1st Qu.:2.525   1st Qu.:4.00   1st Qu.:1.200   versicolor:50  
 Median :5.900   Median :2.800   Median :4.35   Median :1.300   virginica : 0  
 Mean   :5.936   Mean   :2.770   Mean   :4.26   Mean   :1.326                  
 3rd Qu.:6.300   3rd Qu.:3.000   3rd Qu.:4.60   3rd Qu.:1.500                  
 Max.   :7.000   Max.   :3.400   Max.   :5.10   Max.   :1.800                  
-------------------------------------------------------------- 
iris$Species: virginica
  Sepal.Length    Sepal.Width     Petal.Length    Petal.Width          Species  
 Min.   :4.900   Min.   :2.200   Min.   :4.500   Min.   :1.400   setosa    : 0  
 1st Qu.:6.225   1st Qu.:2.800   1st Qu.:5.100   1st Qu.:1.800   versicolor: 0  
 Median :6.500   Median :3.000   Median :5.550   Median :2.000   virginica :50  
 Mean   :6.588   Mean   :2.974   Mean   :5.552   Mean   :2.026                  
 3rd Qu.:6.900   3rd Qu.:3.175   3rd Qu.:5.875   3rd Qu.:2.300                  
 Max.   :7.900   Max.   :3.800   Max.   :6.900   Max.   :2.500                  
\end{lstlisting}

Il risultato è ben diverso. La funzione \textsf{by} ha applicato la funzione \textsf{summary} a tutto il data frame iris dividendolo secondo i fattori presenti nella colonna da noi indicata, nell'esempio Species, restituendo un oggetto sempre di classe by che contiene le statistiche descrittive fornite da \textsf{summary} per tutte le colonne di iris. 

Questa differenza fa si che in molte applicazioni, soprattuto in analisi che comportano l'applicazione di modelli stocastici, la funzione \textsf{by} sia a volte da preferire ma spesso ``l'unica'' via possibile per ottenere il risultato desiderato\footnote{Si notino le virgolette; Per quanto non dimostrabile come affermazione è palese fin dai primi utilizzi che in \erre\ il detto che tutte le strade portano a Roma è vero più che mai. È molto difficile se non quasi impossibile non trovare un modo alternativo di ottenere ciò che si vuole utilizzando altre funzioni o strumenti di \erre. Si rammenti inoltre la vastissima e mondiale comunità di sviluppatori, appassionati, statistici ecc. che scrivono pacchetti.}. 

\subsection{Aggregate}

Alla funzione aggregate è dedicato questo sotto-paragrafo che è tale non per una minore importanza dell'argomento, ma semplicemente perché aggregate può essere interpretata da un punto di vista concettuale come una sorta di tapply potenziata. Sebbene il suffisso apply non sia presente nel nome di questa funzione, essa è da considerarsi a pieno titolo appartenente a tale famiglia perché al suo interno applica tapply. Ne consegue che aggregate potrebbe essere usata in luogo di tapply. I risultati sarebbero gli stessi ma il formato di output sarebbe leggermente diverso.

Si osservi il seguente esempio costruito usando il database iris presente di default in \erre\ nel pacchetto datasets\footnote{Si usi data() per vedere tutti dataset preinstallati in \erre. }
\begin{lstlisting}	
> prova <- tapply(iris$Sepal.Length, iris$Species, mean)
> prova
    setosa versicolor  virginica 
     5.006      5.936      6.588 
>
>
> prova1 <- aggregate(iris$Sepal.Length, list(iris$Species), mean)
> prova1
     Group.1     x
1     setosa 5.006
2 versicolor 5.936
3  virginica 6.588
>
>
> is.atomic(prova)
[1] TRUE
> is.atomic(prova1)
[1] FALSE
>
>
> class(prova)
[1] "array"
> class(prova1)
[1] "data.frame"
\end{lstlisting}

Come si può notare dal codice le due funzioni presentano gli stessi risultati sotto forma diversa. tapply restituisce un array mentre aggregate un data frame. Tuttavia la vera differenza, che si spera non sia sfuggita ad un lettore attento, è negli argomenti delle due funzioni.

La funzione aggregate ha come secondo argomento, l'indice il base al quale suddividere che abbiamo visto in tapply, un oggetto di tipo list. Da questa caratteristica deriva la vera potenza di aggregate: poter applicare una determinata funzione non già su una determinata caratteristica ma bensì su di una lista di caratteristiche. Vediamo un esempio.

Supponiamo di voler calcolare la media di Sepal.Length sulla base della specie, dalla dalla colonna Species ed anche sulla base della variabile Petal.Width. Il codice seguente permette di fare questo. L'unico elemento necessario è quello di usare la funzione list per indicare per quali elementi l'oggetto della nostra funzione, la colonna Sepal.Length, deve essere suddiviso.
\begin{lstlisting}	
> prova5 <- aggregate(iris$Sepal.Length, list(iris$Species, iris$Petal.Width), mean)
> head(prova5)
  Group.1 Group.2        x
1  setosa     0.1 4.820000
2  setosa     0.2 4.972414
3  setosa     0.3 4.971429
4  setosa     0.4 5.300000
5  setosa     0.5 5.100000
6  setosa     0.6 5.000000
> tail(prova5)
     Group.1 Group.2        x
22 virginica     2.0 6.650000
23 virginica     2.1 6.916667
24 virginica     2.2 6.866667
25 virginica     2.3 6.912500
26 virginica     2.4 6.266667
27 virginica     2.5 6.733333
>
>
> class(prova5)
[1] "data.frame"
\end{lstlisting}

Piccola postilla in conclusione di questo paragrafo. Lo stesso risultato in verità si potrebbe ottenere con la stessa funzione tapply, tuttavia il risultato sarebbe stato un oggetto poco pratico da usare. Si provi il seguente codice.
\begin{lstlisting}	
prova3 <- tapply(iris$Sepal.Length, list(iris$Species, iris$Petal.Width), mean)
\end{lstlisting}

\section{mapply}

Da un punto di vista concettuale \textsf{mapply} potrebbe essere quella più tricky. La forza di \textsf{mapply} sta nel fatto di poter essere applicata ad elementi multipli. Il funzionamento è il seguente: dati due (o più) oggetti \erre, ad esempio due vettori di uguale lunghezza, \textsf{mapply} estrae il primo elemento di ognuno e vi applica la funziona specificata, dopodiché estrae il secondo elemento di ogni oggetto e vi applica la funzione, dopodiché estrae il terzo elemento di ogni oggetto e vi applica la funzione\dots\ e così via fino a che tutti gli elementi di tutti gli oggetti non sono stati trattati.

Come al solito un esempio può essere chiarificatore.

Supponiamo di avere due vettori p1 e p2 e di voler sommare il primo elemento di p1 con il primo di p2, il secondo di p1 con il secondo di p2 ecc.

\begin{lstlisting}
> p1 <- c(1:5)
> p2 <- c(10:14)
> mapply( "+" , p1, p2)
[1] 11 13 15 17 19
\end{lstlisting}

Considerazioni sull'esempio:
\begin{itemize}
\item l'esempio è volutamente banale; lo stesso risultato si sarebbe potuto avere semplicemente con \textsf{> p1 + p2}, alcuni esempi più complessi si avranno nei successivi paragrafi\footnote{Oltre che banale questo esempio in realtà è anche, volutamente e per fini didattici, sbagliato. Esso infatti non sfrutta la cosiddetta vettorizzazione delle operazioni cosa che per grandi quantità di dati è decisamente più efficiente da un punto di vista del tempo necessario ad effettuare le operazioni. Non si sottovaluti questo aspetto nell'era dei big data.}.
\item si noti l'uso delle virgolette nello specificare la funzione somma.
\item il semplice operatore di addizione, il simbolo ``+'', è in realtà una funzione.
\end{itemize}

\subsection{Introduzione alle anonymous functions}

Al fine di meglio comprendere la funzione \textsf{mapply} fornendo esempi più complessi, si introducono brevemente le anonymous functions molto usate dalle funzioni della famiglia \textsf{apply}. Le anonymous functions sono funzioni che non hanno un nome perché definite solo all'interno della funzione (\textsf{mapply}, ma anche \textsf{apply}, \textsf{sapply} ed altre) che di esse si serve ed usate solo al suo interno. Una volta eseguite, sostanzialmente spariscono.

Un esempio semplice è rappresentato dal seguente codice che sostanzialmente istruisce \erre\ su una funzione composta da due argomenti x ed y che restituisce come output un valore dato dalla potenza del primo, x, elevato alla seconda, y.

\begin{lstlisting}
function(x,y) x^y
\end{lstlisting}


\subsection{mapply: esempi con le anonymous functions}

Ora che brevemente sono state introdotte le anonymous functions è possibile sfruttarle per costruire qualche esempio più complesso su come utilizzare la funzione \textsf{mapply}.

Supponiamo di avere tre vettori di eguale lunghezza 
\begin{lstlisting}
> p1 <- c(1:5)
> p2 <- c(10:14)
> p3 <- c(-1, 1, -1, 2, 1)
\end{lstlisting}

Usiamo questi vettori e la funzione \textsf{mapply} per ottenere quel valore dato dalla quoziente avente al numeratore il prodotto del primo elemento di p1, moltiplicato il primo elemento di p2 e come denominatore il primo elemento di p3.

\begin{lstlisting}
> mapply( function(x, y, z) (x*y)/z , p1, p2, p3)
[1] -10  22 -36  26  70
\end{lstlisting}

Una simile funzione in \erre\ non esiste ecco perché si è reso necessario definirla all'interno della funzione \textsf{mapply}.

Un altro esempio illustrativo potrebbe essere il seguente che fa uso della funzione \textsf{rep(<valore>, <volte>)}, che sostanzialmente ripete il <valore> un numero di <volte>.
\begin{lstlisting}
> mapply( function(x,y,z){ rep(y, x) * z } , p1, p2, p3)
[[1]]
[1] -10

[[2]]
[1] 11 11

[[3]]
[1] -12 -12 -12

[[4]]
[1] 26 26 26 26

[[5]]
[1] 14 14 14 14 14
\end{lstlisting}

Come prima la funzione anonima function ha tre elementi. Essa è istruita per prendere il primo elemento di p1, ripeterlo un numero di volte pari al numero del primo elemento p2 e moltiplicare questo output per il primo elemento di p3. Così per tutti gli elementi i-esimi dei tre vettori di eguale lunghezza.

\section{Procedura Split-Apply-Combine}

Le funzioni viste fino ad ora possono essere inquadrate come parte di una procedura più complessa denominata split-apply-combine.
