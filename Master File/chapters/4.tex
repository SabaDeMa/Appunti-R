\chapter{Manipolare Stringhe}
\section{Stringhe}
Una parte importante che 
Un importante passo per la pulizia,  dei dati è manipolare stringhe di testo.  
Una prima funzione semplice che potrebbe tornare utile nella manipolazione di stringhe è \textsf{tolower} che permette di eliminare caratteri maiuscoli da un   oggetto di classe character datogli come argomento. 
\begin{lstlisting}
> x <- c("prova", "prOVA3")
> tolower(x)
[1] "prova"  "prova3"
> x
[1] "prova"  "prOVA3"
\end{lstlisting}
Si badi bene, e questo è chiaro con la riga, che \textsf{tolower} restituisce gli elementi dell'oggetto datogli come argomento, ma non non li sovrascrivere. Per ottenere ciò dobbiamo assegnare alla funzione un nuovo oggetto o sovrascrivere quello che già abbiamo.
\begin{lstlisting}
> x <- tolower(x)
> x
[1] "prova"  "prova3"
\end{lstlisting}

Un'altra funzione è \textsf{strsplit} che permette di dividere (da qui lo split nel nome) stringhe sulla base del secondo argomento. Eccone un esempio:
\begin{lstlisting}
> x <- c("prova.1", "prova.2", "prova.3")
> x <- strsplit(x, "\\.")
> x
[[1]]
[1] "prova" "1"    

[[2]]
[1] "prova" "2"  
\end{lstlisting}
Alcune considerazioni:
\begin{itemize}
\item anche in questo caso la funzione non opera sull'oggetto che rimane come definito dall'utente. Per sostituirlo è necessario o crearne un altro oppure, come nell'esempio, sovrascriverlo.
\item si noti che l'output è un oggetto list
\item si noti che l'intento era quello di dividere in due le stringhe sulla base del punto contenuto al loro interno. Nella applicazione della funzione tuttavia se avessimo usato solo il punto \erre\ avrebbe restituito un errore in quanto il punto è il separatore decimale; sono necessari quindi i due``\textbackslash\textbackslash'' per fare ciò che tecnicamente si chiama ``escape'', ossia istruire il programma ad ignorare momentaneamente un metacarattere\footnote{Un simbolo che, per il software utilizzato, ha un significato particolare}.
\end{itemize}


È possibile rimuovere alcuni simboli da stringhe e sostituirli con un altri utilizzando la funzione \textsf{sub} che necessita di tre argomenti. La sua sintassi è la seguente:
\begin{lstlisting}
sub("<simbolo>", "", names(<oggetto>))
\end{lstlisting}
Questo comando dice di sostituire con \textbf{niente}, per questo il secondo argomento è composta da virgolette senza nessun simbolo (neanche uno spazio che di fatti è un simbolo!), ogni volta che incontra, all'interno dei nomi di <oggetto>, un underscore <simbolo>.

Di seguito un esempio in cui si sostituisce la chiocciola con nessun simbolo.
\begin{lstlisting}
> x <- c("dati@1", "dati@2", "dati@3")
> x <- sub("@", "", x)
> x
[1] "dati1" "dati2" "dati3"
\end{lstlisting}

Ipotizzando ora che all'interno della stessa stringa uno stesso simbolo appaia più volte, la funzione \textsf{sub} permette di eliminare solo il primo di questi. Per eliminare tutti i simboli, e non solo il primo, bisogna usare la funzione \textsf{gsb}.

L'esempio seguente mostra il suo funzionamento, simile a \textsf{sub}.

\begin{lstlisting}
> sub("@", "", "prova@CON@più@SIMBOLI")
[1] "provaCON@più@SIMBOLI"
> gsub("@", "", "prova@CON@più@SIMBOLI")
[1] "provaCONpiùSIMBOLI"
\end{lstlisting}


\section{Ricerca di Stringhe}
Posso cercare tra i nomi tutte le stringe che contengono una determinata parola:
\begin{lstlisting}
	grep("<parola>", <oggetto>)
\end{lstlisting}
Grep restituisce un vettore con le posizioni degli elementi dell’oggetto che contengono la parola ricercata. Per far si che ritorni i valori bisogna settare l’opzione value = TRUE.

Qualora invece fosse più comodo ai nostri fini ottenere un vettore logico è possibile utilizzare la funzione grepl che funziona allo stesso modo di grep 
Posso usare anche grepl che si differenzia da grep perché restituisce un vettore logico di lunghezza pari a quella dell’oggetto cercato composto da elementi TRUE qualora il valore cercato fosse presente alla i-esima posizione ed ovviamente FALSE in caso contrario.

Una molto utile combinazione di funzioni per la ricerca di stringhe è la seguente:
\begin{lstlisting}
length(grep(“<parola>”, <oggetto>))
\end{lstlisting}

Questa funzione restituisce un numero pari a quello delle stringhe <parola> presenti nell’oggetto.

\subsection{Varie}
Di seguito vengono introdotte brevemente alcune funzioni varie per lavorare con stringhe. Per dettagli su ognuna di esse si veda la relativa documentazione.
\begin{description}
\item[nchar] restituisce il numero di caratteri in una stringa; si noti che uno spazio vuoto è un carattere.
\item[paste] permette di fondere insieme due stringhe; per unire senza spazio è possibile usare anche la variante paste0 che come specificato nella documentazione, è più efficiente.
\item[substr] estrae da una stringa (o un vettore di stringhe) gli elementi i-esimi fino al j-esimo. Nulla vieta di estrarne solo uno, in questo caso 
\end{description}

un'altra funzione utile può essere nchar() che mi dice il numero di caratteri che ci sono in una stringa; oppure substr(<stringa>, 1, 7) che mi prende solo le lettere dalla uno alla sette della <stringa>.

paste() permette di fonderle insieme fuse da uno spaio, se metto solo le stringhe oppure unite da un altro simbolo se lo imposto con 

Posso fonderle senza spazio usando direttamente la funzione paste0()

La funzione str_trim del pacchetto stringr permette di eliminare lo spazio in eccesso.
	> str_trim("Prova      ")
	[1] "Prova"


\section{Regular Expressions}

Metacaratteri permettono di eseguire ricerche...

^<parola>  questo mi cerca tutte le parole, frasi, elementi che INIZIANO con <parola>
$<parola>  questo mi cerca tutte le frasi che finiscono con <parola>
^[Ii] am   questo mi cerca tutte le frasi che iniziano con i/I am

Posso anche mettere range
^[0-9][a-zA-Z]  questo mi cerca tutte le frasi che iniziano con un numero compreso tra zero ed uno e che sono seguite da una lettera compresa tra a e z e tra A e Z.

Un altro metacarattere è il semplice punto . .

	9.11 ad esempio mi cerca tutte le combinazioni di stringhe che hanno un nove seguito da un simbolo qualsiasi da poi da un 11. Il punto quindi ha valore di jolly valendo in questo caso qualunque simbolo purché, nell'esempio, preceduto da un nove e seguito da un 11.

Un altro carattere è | . Questo cercherà tutte le stringhe che hanno l'una o l'altra espressione:  <parola1>|<parola2> mi cerca si c'è <parola1> oppure se c'è <parola2> e non se ci sono entrambe anche se ti traduce con "and". Non ci vogliono spazi, come negli altri metacaratteri, altrimenti R si blocca.

Ne posso mettere pure due o tre di  |||


Fare attenzione alle sottigliezze 

^([Gg]ood|[Bb]ad)
^[Gg]ood|[Bb]ad

L'unica differenza è data dalle parentesi tonde ma il comportamento di queste due espressioni è molto diverso. La prima cercherà frasi che iniziano o per good/Good oppure che iniziano per bad/Bad. La seconda invece cercherà frasi che iniziano per good/Good oppure che hanno al suo interno bad/bad.

Un punto esclamativo indica che l'espressione cercata è opzionale.

Da notare che un metacarattere è un carattere che per un programma ha un particolare significato. vedere wiki http://en.wikipedia.org/wiki/Metacharacter.


* indica qualsiasi numero di volte

ad esempio (.*) cerca qualsiasi parenti che abbia al suo interno qualsiasi carattere un numero qualsiasi di volte quindi mi troverà anche solo ().

Il segno più include almeno uno dei simboli

[0-9]

\subsection{Metacaratteri}

Nella ricerca di particolari stringhe quasi sempre però gli obiettivi sono molto più complessi di quelli visti negli esempi precedenti. Potrebbe capitare di dover seleziona tutte le stringhe che possiedano al loro interno un determinato carattere o combinazioni di più caratteri, oppure qualsiasi stringa che inizi con un determinato simbolo, o che con questi vi termini. Si deduce da queste semplici considerazioni che sono necessari strumenti per istruire meglio e nel giusto modo le funzioni in erre, ma anche in altri linguaggi, per estrarre stringhe secondo le esigenze dell’utente.

Questi strumenti sono i metacaratteri. I metacaratteri sono caratteri che all’interno di un linguaggio di programmazione assumono un particolare significato. \erre\ utilizza i 14 metacaratteri del POSIX extended regular expressions di cui si parlerà nel prosieguo di questo paragrafo. Ovviamente fin da ora è bene specificare che i metacaratteri, in quanto tali, hanno un significato proprio in \erre\ ed in molti altri linguaggi che supportano questo standard ma allo stesso tempo sono grafi utilizzati nella comune scritta.

Si pensi ad esempio ai simboli delle parentesi quadre [ ].  Nella comune compilazione di testi, secondo le odierne consuetudini tipografiche queste sono utilizzate per “per inserire delle correzioni, specialmente nelle citazioni; delimitano, insomma, delle parti che nella citazione originale mancavano o erano diverse” come affermato in~\cite{latex:guida}. Il loro significato come metacaratteri è molto diverso in quanto permette di creare quella che in \erre\ è definita come una classe di caratteri che è più che altro un nome per identificare un contenitore come si vedrà e \textbf{non} ha nulla a che vedere con la modalità character propria delle classi di \erre.

Per utilizzare uno od entrambi i simboli delle parentesi quadre nel loro significato tipografica è necessario quindi far si che l’interprete ignori il metacarattere e lo consideri un semplice glifo. In \erre\ tutti i metacaratteri sono ignorati e quindi considerati secondo il loro proprio significato tipografico con due backslash.

La tabella seguente, tratta da ~\cite[san.gast] riassume i 14 metacaratteri ed il codice da utilizzare per ignorarli come tali e considerali come semplici glifi.

\begin{tabular}{ll}
\toprule
metacarattere	& significato \\
\midrule
\textbf{\^a}	& seleziona tutti gli elementi che iniziano con la lettere a.\\
\textbf{a\$} 	& seleziona tutti gli elementi che terminano con a. Da notare che il simbolo va posto successivamente al carattere da ricercare.\\
\textbf{.}		& il punto fermo indica una selezione totale dei caratteri, vale a dire è un selettore globale di tutti i tipi di caratteri. Si noti ad esempio il  seguente utilizzo 10.a che selezionerà tutte le stringhe che possiedono 
\bottomrule
\end{tabular}













