\chapter{Programmazione}

\section{Costrutti}
Quando le funzioni della famiglia \textsf{apply} viste in precedenza non sono sufficienti a raggiungere i nostri scopi si rende necessario l'utilizzo di strumenti più sofisticati. Si consideri che questa esigenza nella programmazione di script o nella creazione di funzioni è molto più frequente di quanto si possa pensare.  Gli strumenti a cui si faceva riferimento prima sono i \textsf{loops} o per essere più precisi, costrutti per il controllo del flusso\footnote{\textsf{Loops} è decisamente un nome più digeribile.}.
Le strutture di controllo permettono sostanzialmente di controllare il flusso di come le operazioni di un programma vengono eseguite. Strutture comuni sono:

\begin{description}
\item[if,else] servono per testare una condizione
\item[ifelse] controlla una condizione ed esegue in base al risultato una specifica operazione
\item[for] esegue un loop un numero finito di volte
\item[while] esegue un loop fino a quando la condizione specificata è rispettata
\item[repeat] lo ripete in infinito
\item[break] (inter)rompe l'esecuzione di un loop, necessario con repeat
\item[next] salta l'interazione di un loop
\item[return] permette di uscire da una funzione 
\item[switch] valuta una affermazione e restituisce una operazione tra una lista di operazioni, basata sul risultato della affermazione
\end{description}

\subsection{STRUTTURA IF-ELSE}

Il costrutto \textsf{if} permette di far eseguire ad \erre\ una o più operazioni determinate e definite tra le parentesi graffe, sulla base della veridicità(\textsf{TRUE}) o meno (\textsf{FALSE}) di una o più condizioni definite tra le parentesi tonde. La sintassi può essere interpretata anche da un punto di vista letterale come ``se è vero questo\dots, esegui questa operazione'', oppure anche ``esegui questa operazione se è vero che\dots''.

Il costrutto \textsf{else} che è opzionale e diventa operativo solo se la condizione non è rispettata, ovvero se è \textsf{FALSE}, permette di eseguire, qualora lo si voglia, un'altra operazione diversa dalla prima dichiarata con \textsf{if}. Ci sono diversi modi di formulare costrutti logici usando questi due strumenti.
Si consideri il seguente esempio.

\begin{lstlisting}
> x <- 3
> if(x > 1) {
+ 	print("x è maggiore di uno")
+ } else {
+ 	# Non fa niente
+ }
[1] "x è maggiore di uno"
\end{lstlisting}
È possibile anche fare in un altro modo:
\begin{lstlisting}
y<- if(x>3) { 10 } else {0}
\end{lstlisting}

else è opzionale ma la si può mettere se voglio fare qualcosa di alternativo ad if, ma si può anche omettere, oppure fare un intreccio
di if conditions:
\begin{lstlisting}
if(<conditions>) {}
if(<conditions>) {}
\end{lstlisting}

\subsection{Costrutto for}

È molto comune, l'idea di base è di avere un indice spesso chiamato i che va tipicamente
come una sequenza di numeri interi.
Un esempio banale è questo che in pratica mi stampa i numeri 1:10
(quindi da uno a dieci), la variabile è i:

\begin{lstlisting}
for(i in 1:10) {print(i)}
\end{lstlisting}

Una volta terminato il loop, si passa al blocco di codice successivo ad esso.
Ci sono diversi modo di usare un loop for.
Questi modi sotto sono tutti equivalenti e mi stampano le lettere dell'oggetto c.

\begin{lstlisting}
x<-c("a","b","c","d")
for(i in 1:4) {print(x[i])}

for(i in seq_along(x)) { print(x[i])} 
\end{lstlisting}

seq\_along prende un vettore come input e crea una sequenza di interi
che è uguale alla lunghezza di questo vettore, che quindi è lo stesso di 1:4 in questo caso.
Nel caso successivo chiamo la variabile indice letter perché non necessariamente deve esser un intero ma può prendere i valori da qualsiasi vettore:

\begin{lstlisting}
for(letter in x) {print(letter)}
\end{lstlisting}

L'ultimo è esattamente come gli altri anche perché in effetti dopo in ho messo x e quindi quando do il comando letter (ma avrei potuto chiamarlo in qualsiasi modo) in effetti dico ad \erre di non far altro che andarsi a prendere letter che in realtà è x in questo caso e di stamparlo.

Quando la condizione è una come in questo caso posso ometterle e scrivere in forma compatta il loop\footnote{Ma meglio evitare\dots}.

\begin{lstlisting}
for(i in 1:4) print(x[i])
\end{lstlisting}

I for loops possono essere concatenati.
Un esempio banale può essere questo:

\begin{lstlisting}
x<-matrix(1:6,2,3)

for(i in seq_len(nrow(x))) {
	for(j in seq_len(ncol(x))) {
		print(x[i,j])
	}
}
\end{lstlisting}

Prima creo una matrice con il comando matrix; poi uso la funzione funzione seq\_len che prende un numero intero del suo argomento che in questo caso sono il numero delle righe della matrice x (date dalla funzione nrow()) e ne crea una sequenza. Quindi in effetti l'argomento è nrow(x). Che di fatto è un numero e da questo ne crea
una sequenza di interi. In questo caso avendo due righe la sequenza che ne esce fuori è
1 e 2.
Quando invece l'argomento è ncol(x) la sequenza è 1,2,3.
Alla fine questo comando mi dice di stampare tutti gli elementi della matrice.
Il comando essenzialmente è quello di stampare l'elemento di riga i e colonna j preso
dalla matrice x infatti i va da 1 a 2, e j va da 1 a 3 quindi mi stampa sei elementi, i sei della matrice.

Meglio evitare di innestare troppi loop perché benché funzionante si potrebbe creare un comando molto difficile da interpretare. Spesso questa complessità può essere evitata semplicemente ricorrendo alla scrittura di una funzione.

\subsection{while}

L'idea di base è che \erre prende quel costrutto logico ed esegue le operazioni basandosi su questo.
In pratica, se la condizione è rispettata, \erre esegue la(e) operazione(i) e la(e) esegue fino a quando sussiste(-ono). Questo però
fa si che il loop potrebbe essere teoricamente infinito se la condizione che lo fa fermare
non esiste, altrimenti il programma non finirà mai\footnote{Un esempio di loop infinito è while(count<10)\{print(count)\} }.
Esempio (che non funziona!):

\begin{lstlisting}
count<-0
while(count <10) {print(count) count<- count +1}
\end{lstlisting}
Io ho provato questa e funziona alla grande:
\begin{lstlisting}
count<-0
while(count<10){print(count<-count+1)
\end{lstlisting}

Ovviamente si possono inserire più condizioni in un solo loop.
In questo esempio la variabile z è inizializzata a 5. La condizione while è che sia maggiore o uguale di 3 e minore o uguale di 10:
\begin{lstlisting}
z<-5

while(z>=3 && z<=10) {
		      print(z)
		      coin<-rbinom(1,1,0.5)
		      if(coin==1) {## RANDOM WALK
		      z<-z+1} else {z<-z-1}
}
\end{lstlisting}

Dove la funzione rbinom(x,size,pro) è una funzione di densità, distribuzione,
generazione di numeri casuali di una binomiale con x è il vettore dei quantili,
size è il numero delle prove da effettuare e prob è la probabilità\footnote{Ci vorrebbe
un ripasso della binomiale in effetti!}.
Quando \erre si trova a dover verificare diverse condizioni, segue uno schema
logico interno: le verifica partendo da sinistra verso destra.
Nell'esempio in effetti prima vede se z>=3 e una volta verificata questa 
verifica anche la seconda che z<=10.
Se entrambe sono vere, finalmente \erre va avanti e va nel corpo del loop while
ed esegue i comandi in esso specificati.


\subsection{REPEAT, NEXT, BREAK}

REPEAT

Repeat non è molto frequente perché inizializza un loop infinito e l'unico modo per fermarlo è invocare ad un certo punto un break.
Nell'esempio prima definisco x0 e tol.

\begin{lstlisting}
x0<-1
tol<-1e-8

repeat{
	x1<-computeEstimate() #questa funzione non esiste! solo per esempio!
	if(abs(x1-x0)<tol) {break}
	else{x0<-x1}
}

\end{lstlisting}
In pratica questo esempio dice che fissa x1 come il risultato di una funzione, in questo caso generica che non esiste, dopodiché se il valore assoluto della differenza tra x1 (risultato di questa funzione fittizia) è minore di tol si ferma altrimenti imposta x0 come x1 e ripete il loop dall'inizio.
Questo loop è pericoloso perché non è detto che si fermi quindi è meglio usare un for
loop ed usare un limite di iterazioni e alla fine osservare la convergenza.

\subsection{NEXT}

Next è usato per skip (saltare) una iterazione di un loop (di qualsiasi tipo). Un esempio semplice è questo:

\begin{lstlisting}
for(i in 1:100) {if(i <= 20) {next}
		fa_qualcosa_qua() # esempio
}
\end{lstlisting}


\subsection{Switch}





